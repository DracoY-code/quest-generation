\clearpage

\section*{\centering \large ABSTRACT}
\addcontentsline{toc}{section}{Abstract}

\vspace{2em}

\begin{doublespace}
  \justifying
  \noindent
  This project investigates the use of large language models (LLMs) for procedural quest
  generation (PQG) in role-playing games (RPGs) to enhance narrative diversity and reduce
  manual content authoring. Procedural content generation has long been employed in game
  design to create expansive and dynamic game worlds, but narrative elements—particularly
  quests—remain challenging to automate due to their reliance on coherence, structure, and
  context. To address this, we utilize modern transformer-based LLMs capable of producing
  contextually rich text based on structured input representations. The dataset used
  in this project is derived from publicly available quest data, formatted into a consistent
  XML-inspired schema with clearly defined tags representing quest components such as
  objectives, tasks, characters, rewards, and locations. This structured format allows the
  models to focus on learning narrative patterns while preserving semantic alignment with
  the game world. To enable efficient training on limited hardware, we adopt a two-stage
  fine-tuning approach. First, the models—GPT-2, LLaMA 3.2, and TinyLlama—are quantized
  using 4-bit techniques to reduce memory consumption. Then, parameter-efficient
  fine-tuning (PEFT) is performed using Low-Rank Adaptation (LoRA), which injects trainable
  adapters into selected model layers, significantly lowering the number of trainable
  parameters while preserving performance. A custom pipeline is developed to manage the
  tokenization, quantization, and LoRA integration, enabling fine-tuning on constrained setups.
  The quality of generated quests is evaluated using both automatic metrics—BLEU,
  METEOR, BERTScore, and perplexity—and manual inspection for narrative consistency
  and fluency. Results show that larger LLaMA models generally outperform GPT-2 variants
  in producing structured and coherent quests, although trade-offs exist in training
  time and resource usage. This project demonstrates that with minimal supervision and
  efficient adaptation techniques, LLMs can effectively generate believable and diverse RPG
  quests. The findings lay the groundwork for scalable narrative generation systems that
  could be integrated into modern game engines and development pipelines.
\end{doublespace}

\newpage
